\documentclass{article}
\usepackage[margin=0.7in]{geometry}
\usepackage[titletoc,title]{appendix}

\title{Wiederholung Bewegungslehre Klasse 10}
\author{September 2022}
\date{}


\begin{document}
\maketitle
\section*{Newtonsche Grundgesetze}
\subsection*{1. Trägheitsgesetz}
\begin{itemize}
	\item Ein ruhender Körper bleibt in Ruhe, wenn keine äußeren Kräfte auf ihn einwirken.
\end{itemize}
\subsection*{2. Aktionsprinzip}
\begin{itemize}
	\item Wirkt auf einen Körper eine resultierende Kraft $\vec{F}$, so wird der Körper in die Richtung der Kraft beschleunigt.
\end{itemize}
\begin{center}
	$\vec{F} = m \cdot \vec{a} = m \cdot \frac{\Delta \vec{v}}{\Delta t}$
\end{center}
\subsection*{3. Actio - Reactio}
\begin{itemize}
	\item Übt $A$ eine Kraft (\textit{actio}) auf $B$ aus, so übt $B$ eine gleich große, entgegengesetzt gerichtete Kraft (\textit{reactio}) auf $A$ aus.
\end{itemize}
\subsection*{Superpositionsprinzip}
\begin{itemize}
	\item Verschiedene Kräfte, die alle einzeln auf den gleichen Körper wirken, bewirken dasselbe, als würde lediglich ihre Summe auf den Körper wirken. Dh. solange sich Kräfte nicht behindern können sie überlagert werden.
\end{itemize}
\section*{Geradlinige Bewegungen}
\end{document}