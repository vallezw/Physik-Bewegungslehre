\documentclass{article}
\usepackage[margin=0.5in, bottom=0.8in]{geometry}
\usepackage[titletoc,title]{appendix}

\title{Wiederholung Bewegungslehre Klasse 10}
\author{September 2022}
\date{}


\begin{document}
\maketitle
\section*{Newtonsche Grundgesetze}
\subsection*{1. Trägheitsgesetz}
\begin{itemize}
	\item Jeder Körper beharrt in seinem Zustand der Ruhe oder der gleichförmigen Bewegung mit konstanter Geschwindigkeit, wenn der Körper nicht durch einwirkende Kräfte gezwungen wird, seinen Zustand zu ändern.	
\end{itemize}
\subsection*{2. Aktionsprinzip}
\begin{itemize}
	\item Wirkt auf einen Körper eine resultierende Kraft $\vec{F}$, so wird der Körper in die Richtung der Kraft beschleunigt.
\end{itemize}
\begin{center}
	$\vec{F} = m \cdot \vec{a} = m \cdot \frac{\Delta \vec{v}}{\Delta t}$
\end{center}
\subsection*{3. Actio - Reactio}
\begin{itemize}
	\item Übt $A$ eine Kraft (\textit{actio}) auf $B$ aus, so übt $B$ eine gleich große, entgegengesetzt gerichtete Kraft (\textit{reactio}) auf $A$ aus.
\end{itemize}
\subsection*{Superpositionsprinzip}
\begin{itemize}
	\item In der Physik versteht man darunter eine Überlagerung gleicher physikalischen Vektorgrößen, wobei sich jene nicht gegenseitig behindern (ungestörte Überlagerung). Die Addition der jeweiligen Größen erfolgt dann vektoriell.
\end{itemize}
\section*{Geradlinige Bewegungen}
\subsection*{Gleichförmige Bewegung}
\begin{itemize}
	\item Bei einer gleichförmigen Bewegung ist die Geschwindigkeit, sowie die Richtung konstant und ändert sich \textbf{nicht}.
\end{itemize}
\begin{center}
	$s(t) = v \cdot t + s_0$ ; $v(t) = v_0 =$ konstant
\end{center}
\subsection*{Gleichmäßig beschleunigte Bewegung}
\begin{itemize}
	\item Bei der gleichmäßig beschleunigten Bewegung handelt es sich um eine Bewegung, deren Beschleunigungskraft konstant ist.
\end{itemize}
\begin{center}
	$s(t) = \frac{1}{2}a \cdot t^2 + v_0 \cdot t + s_0$\\
	$v(t) = a \cdot t + v_0$\\
	$a \neq 0$ ; $\vec{a} =$ konstant
\end{center}
\section*{Freier Fall}
\begin{itemize}
	\item Auf den Körper wirkt nur seine eigene Gewichtskraft $\vec{F}_G$
	\item Es gelten die Bewegungsgesetze der gleichmäßig-beschleunigten Bewegung mit $a=g$
\end{itemize}
\begin{center}
	$y(t) = -\frac{1}{2} \cdot g \cdot t^2 + y_0$\\
	$v_y(t) = -g \cdot t$\\
	$a_y(t) = -g$\\
	$t_{Fallzeit} = \sqrt{\frac{2 \cdot y_0}{g}}$
\end{center}
\section*{Waagerechter Wurf}
\begin{itemize}
	\item Die Bewegungen in $x$- sowie $y$-Richtung beeinflussen sich laut dem Superpositionsprinzip nicht, solange Reibungseffekte vernachlässigt werden.
	\item In $x$-Richtung bewegt sich der Körper gleichförmig mit $x(t) = v_0 \cdot t$.
	\item In $y$-Richtung bewegt sich der Körper gleichmäßig beschelunigt wie beim freien Fall mit $y(t) = -\frac{1}{2} \cdot g \cdot t^2 + h$.
	\item Die Bahnkurve $y(x)$ ist eine Parabel mit $y(x) = -\frac{1}{2} \cdot \frac{g}{{v_{0}}^2} \cdot x^2 + 2$.
\end{itemize}
\section*{Mechanische Energieformen}
\subsection*{Potentielle Energie}
\begin{itemize}
	\item Die potentielle Energie $E_{pot}$ eines Körpers ist proportional zu seiner Masse $m$ , dem Ortsfaktor $g$ und zur Höhe $h$ des Körpers über einem definierten Nullniveau (meist dem Erdboden).
\end{itemize}
\begin{center}
	$E_{pot} = m \cdot g \cdot h$.
\end{center}
\subsection*{Kinetische Energie}
\begin{itemize}
	\item Die kinetische Energie $E_{kin}$ eines Körpers ist proportional zu seiner Masse $m$ und proportional zum Quadrat $v^2$ seiner Geschwindigkeit.
\end{itemize}
\begin{center}
	$E_{kin} = \frac{1}{2} \cdot m \cdot v^2$
\end{center}
\subsection*{Spannenergie}
\begin{itemize}
	\item Die Spannenergie $E_{spann}$ einer gedehnten Feder ist proportional zu ihrer Federkonstante $D$ und proportional zum Quadrat $s^2$ ihrer Längenänderung.
\end{itemize}
\begin{center}
	$E_{spann} = \frac{1}{2} \cdot D \cdot s^2$
\end{center}
\section*{Energieerhaltungssatz}
\begin{itemize}
	\item In einem reibungsfreien System bleibt die Gesamtenergie \textbf{gleich}, wenn es von außen nicht beeinflusst wird.
	\item Mathematisch kann man die Energieerhaltung ausdrücken als
\end{itemize}
\begin{center}
	$E_{Ges} = E_{pot} + E_{kin} + E_{spann} = $ \textbf{konstant}
\end{center}
\end{document}
