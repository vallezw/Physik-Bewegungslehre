\documentclass{article}
\usepackage[margin=0.5in]{geometry}
\usepackage[titletoc,title]{appendix}

\title{Wiederholung Bewegungslehre Klasse 10}
\author{September 2022}
\date{}


\begin{document}
\maketitle
\section*{Newtonsche Grundgesetze}
\subsection*{1. Trägheitsgesetz}
\begin{itemize}
	\item Ein ruhender Körper bleibt in Ruhe, wenn keine äußeren Kräfte auf ihn einwirken.
\end{itemize}
\subsection*{2. Aktionsprinzip}
\begin{itemize}
	\item Wirkt auf einen Körper eine resultierende Kraft $\vec{F}$, so wird der Körper in die Richtung der Kraft beschleunigt.
\end{itemize}
\begin{center}
	$\vec{F} = m \cdot \vec{a} = m \cdot \frac{\Delta \vec{v}}{\Delta t}$
\end{center}
\subsection*{3. Actio - Reactio}
\begin{itemize}
	\item Übt $A$ eine Kraft (\textit{actio}) auf $B$ aus, so übt $B$ eine gleich große, entgegengesetzt gerichtete Kraft (\textit{reactio}) auf $A$ aus.
\end{itemize}
\subsection*{Superpositionsprinzip}
\begin{itemize}
	\item In der Physik versteht man darunter die Überlagerung gleicher Physikalischer Vektorgrößen, wobei sishc jene nicht gegenseitig behindern. Die Addition der jeweilligen Größen erfolgen vektoriell.
\end{itemize}
\section*{Geradlinige Bewegungen}
\subsection*{Gleichförmige Bewegung}
\begin{itemize}
	\item Bei einer gleichförmigen Bewegung ist die Geschwindigkeit, sowie die Richtung konstant und ändert sich \textbf{nicht}.
\end{itemize}
\begin{center}
	$s(t) = v \cdot t + s_0$ ; $v(t) = v_0 =$ konstant
\end{center}
\subsection*{Gleichmäßig beschleunigte Bewegung}
\begin{itemize}
	\item Bei der gleichmäßig beschleunigten Bewegung handelt es sich um eine Bewegung, deren Stärke sowie Richtung konstant sind.
\end{itemize}
\begin{center}
	$s(t) = \frac{1}{2}a \cdot t^2 + v_0 \cdot t + s_0$\\
	$v(t) = a \cdot t + v_0$\\
	$a \neq 0$ ; $a =$ konstant
\end{center}
\section*{Freier Fall}

\end{document}